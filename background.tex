\chapter{Background}
\label{chap:background}

\paragraph{} We now review some of the necessary background for our work.
We start with a brief introduction to automated planning, one of the earliest areas of Artificial Intelligence, which addresses the problem of synthesizing autonomous behaviors in automated way from a model.
\section{Automated Planning}
\label{background: automated}
Automated Planning is the model-based branch of Artificial Intelligence (AI) focused on obtaining a solution, described through
 a sequence of actions, to a certain task using the knowledge of the world at hand.\\
More specifically, given a model describing the world and the interactions available to agents in it, including their 
corresponding effects on it, we can describe an initial state and a goal state of the world and find a solution that 
transitions between them, if such exists. This solution constitutes a sequence of actions, denoted from now on as a "Plan", 
that if executed accordingly by the domain agents, promises the arrival at the goal state. Finding such plans is not a trivial 
issue and is a task reserved for a family of complex solvers called Planners.\\
Planners are currently seen as automated solvers for precise classes of mathematical models represented in compact form. A typical 
planner takes three inputs: a description of the initial state of the world (initial state), a description of the desired goals (goal state),
 and a set of actions that the executor (agent) is able to perform, all encoded in a formal language such as PDDL \cite{PDDL}. 
Planners are in  general domain-independent in the sense that they do not know what the variables, actions, and domain stand for,
and for any such description they must decide effectively which actions to initiate in order to achieve the goals for the given 
planning task. The planner produces an ordered set of actions that leads from the initial state to a state satisfying all the goals.
\\ 
The actions in the solution produced by the aforementioned planner are sequential and cannot take place in parallel to each other. 
Therefore, it is common to treat these as instantaneous events that start and end immediately, and whose effect on the world occurs at once.
We denote these events as \emph{Instantaneous Happenings} and make use of their properties latter on in the work.

% For the sake of clarity, we will describe a simple planning formalism, STRIPS\cite{STRIPS}, although we make use of the more expressive formalism PDDL $2.1$ further discussed in section \ref{background: Temporal}.

% \input{sections/background_sections/STRIPS}

\paragraph{} In the circumstances of our work, as is in life, it is not enough to have a great plan, there is also a need for timing.  
\subsection{Temporal Planning}
\label{background: Temporal}
Temporal Planning is the extension of the classical planning regimes to the temporal setting.
 The classical formalism does not take into consideration the aspect of time and assumes actions taken in the world are instantaneous and take place in sequential order. In order to model more realistic problems there was need to control when each action should happen and allow actions to occur in parallel to one another. For Instance, the logistics domain would highly benefit from parallel actions if a robot equipped with two arms would be able to use them simultaneously and not one after the other. Adding time to the equation would also allow a factory robot to time his actions according to a shipment scheduled to arrive that same day. When involving time in a solution, the output is no longer a "plan" (sequence of actions) but a "schedule", mapping actions to their designated occurrence and execution stages.
The third International Planning Competition (IPC), held in 2002, presented the planning community with the challenge of handling time. This necessitated the development of a modelling formalism capable of expressing temporal properties of planning domains which lead to the emergence of PDDL2.1

A Temporal Planning problem modeled in the propositional subset of \\ 
PDDL2.1 \cite{fox2003pddl2} is given by a tuple $\Pi = \langle F,A,I,G \rangle$:

\begin{itemize}
    \item $F$: The domain facts. A set of Boolean propositions. The set of all states $S$ is then all the possible subsets of $F$, meaning $\mid S \mid = 2^F$ states.
    \item $A$: The set of durative actions. Each durative action $a \in A$ 
    has a duration $\text{dur}(a) \in [\text{dur}_{min}(a), \text{dur}_{max}(a)]$ and is 
    described by \\
    $a=\langle \text{pre}_{\vdash}(a), \text{eff}_{\vdash}(a),\text{inv}(a),
    \text{pre}_{\dashv}(a), \text{eff}_{\dashv}(a)\rangle$, where:
    \begin{itemize}
        \item Minimum duration $\text{dur}_{min}(a)$ and maximum duration $\text{dur}_{max}(a)$ uphold $0 \leq \text{dur}_{min}(a) \leq \text{dur}_{max}(a)$.
        \item Start condition $\text{pre}_{\vdash}(a) \subseteq F$ (respectively, end condition $\text{pre}_{\dashv}(a)  \subseteq F$), must hold when durative action $a$ starts (respectively, ends).
        \item Start effect $\text{eff}_{\vdash}(a)$ (respectively, end effect $\text{eff}_{\dashv}(a)$), occurs when durative action $a$ starts (respectively, ends). The effects specify which propositions in F become true (add effects), and which become false (delete effects).
        \item Invariant condition $\text{inv}(a) \subseteq F$ which must hold during the whole execution of $a$.
    \end{itemize}
    \item $I$: The initial state, specifying exactly which propositions in $F$ are true at time zero. $I\subseteq F$.
    \item $G$: The goal, which propositions we wish to be true at the end of plan execution. $G \subseteq F$. 
\end{itemize}


% \begin{table}[ht!]
% \centering
% \begin{tabularx}{\textwidth}{ c  X }
% 
% \textbf{Var}  & \textbf{Definition} \\
% \hline
% \\
% $F$:     &  {The domain facts. A set of Boolean propositions. The set of all states $S$ is then all the possible subsets of $F$, meaning $\mid S \mid = 2^F$ states.}  \\
% \\
    % $A$:     & {The set of durative actions. Each durative action $a \in A$ has a duration $\text{dur}(a) \in [\text{dur}_{min}(a), \text{dur}_{max}(a)]$ and is described by $a=\langle \text{pre}_{\vdash}(a), \text{eff}_{\vdash}(a),\text{inv}(a),\text{pre}_{\dashv}(a), \text{eff}_{\dashv}(a)\rangle$, where:
    % \begin{itemize}
        % \item Minimum duration $\text{dur}_{min}(a)$ and maximum duration $\text{dur}_{max}(a)$ uphold $0 \leq \text{dur}_{min}(a) \leq \text{dur}_{max}(a)$.
        % \item Start condition $\text{pre}_{\vdash}(a) \subseteq F$ (respectively, end condition $\text{pre}_{\dashv}(a)  \subseteq F$), must hold when durative action $a$ starts (respectively, ends).
        % \item Start effect $\text{eff}_{\vdash}(a)$ (respectively, end effect $\text{eff}_{\dashv}(a)$), occurs when durative action $a$ starts (respectively, ends). The effects specify which propositions in F become true (add effects), and which become false (delete effects).
        % \item Invariant condition $\text{inv}(a) \subseteq F$ which must hold during the whole execution of $a$.
    % \end{itemize}
    % }      
    % \\
% $I$:     & {The initial state, specifying exactly which propositions in $F$ are true at time zero. $I\subseteq F$.}      \\
% \\
% $G$:     & {The goal, which propositions we wish to be true at the end of plan execution. $G \subseteq F$. }   \\
% 
% \end{tabularx}
% \caption{Temporal Planning PDDL2.1 formalism}
% \label{tab:strips}
% \end{table}

By using the above formalism to describe a temporal planning task, we can commit
$\Pi$ as input to a temporal Planner and obtain a schedule $\tau$.\\

\paragraph{} We now know how to obtain a single solution to our input temporal planning task. If we require a method for obtaining multiple different solutions. 
\subsection{Diverse Planning}
\label{background: Diverse}
Diverse planning is a method for obtaining multiple dissimilar solutions to the same planning task.
 It is commonly used in decision support scenarios where a human analyst needs to understand the space 
of possible plans or has difficulty specifying the planning domain model that exactly matches their application.
The diversity between plans in a plan set $\Lambda$ is measured by the average distance between them.
Previous literature such as \cite{bryce2014landmark} discussed how to best take advantage of plan characteristics 
to compare and measure this distance using domain-independent criteria such as action-sets, states, causal links and landmarks.

% such as shown in Table \ref{tab:diversity measures}.

% \begin{table}[h!]
% \centering
% \begin{tabularx}{\textwidth}{ c  X }

% \textbf{Criteria}  & \textbf{Definition} \\
% \hline
% \textit{Action Sets} & Actions that are present in the two plans.\\
% \\
% \textit{States} & The behaviors resulting from the execution of the plans (where the behavior is captured in terms of the sequence of states the agent goes through).\\
% \\
% \textit{Causal Links} & The causal structures of the two plans measured in terms of the causal links representing how actions contribute to the goals being achieved. \\
% \\
% \textit{Landmarks} & Atomic propositions or disjunctive sets of propositions that must be satisfied by all plans solving a planning instance.\\

% \end{tabularx}
% \caption{Plan Diversity Measures}
% \label{tab:diversity measures}
% \end{table}

Each such criteria has its own distance metric $D(p,p')$ to calculate the dissimilarity between two given plans $p,p'$. 

\begin{tcolorbox}[colback=blue!5!white,colframe=blue!75!black]
For instance, Given two plans $p,p'$ where each plan consists of a sequence of actions,
we can evaluate these to obtain a sequence of states $s_0,s_1..s_n \& s_0',s_1'..s_m$ for each plan. 
The distance between a pair of states can then be computed as: 
\[S(s,s') = \frac{\mid s \Delta s' \mid}{\mid s \cup s' \mid}\]
Where $\Delta$ is the symmetric difference operator.
To account for difference in plan length, two methods are suggested.
The first method assumes $k \leq k'$ and each state $s_{k'+1}..s_k$ is considered to contribute maximally (i.e one unit) 
to the difference between the plans:
\[D_{S}(p,p') = \frac{1}{k} \times \bigg[\sum^k_{i=1} S(s_i,s'_i) + k'-k\bigg]\]
The second method assumes the shorter plan simply stays in the goal state $s_g$ for the rest of the duration of the longer plan:
\[D_{S}(p,p') = \frac{1}{k} \times \bigg[\sum^{k-1}_{i=1} S(s_i,s'_i) + \sum^{k'}_{i=k} S(s_k,s'_i)\bigg] \]

\end{tcolorbox}


\paragraph{The Maximal Plan Diversity Problem}
\textit{Given a plan distance metric $D$, a number $k$ and a planning problem instance $P$, obtain a collection of $k$ solution plans $p_1, p_2...p_k \in \Lambda$ solving $P$, such that no other set of $k$ solution plans $\Lambda'$ solving $P$ upholds $Div(\Lambda) < Div(\Lambda')$ } 
\\
Finding the maximum set can be impossible depending on how $D$ is defined\cite{Coman2011GeneratingMetrics}. Therefore, a relaxation of this problem is solved instead, where a distance metric D is taken into account during plan set generation, but maximal plan diversity is not guaranteed.

    
\paragraph{Top-k Planning}
Top-$k$ planning is the task of obtaining a set of $k$ solutions s.t. there exists no better quality solution outside that set. This approach values the quality of the solutions over their diversity from one another.
The iterative plan Forbid Reformulation \cite{katz2018novel} approach to top-$k$ planning
exploits cost-optimal planners by iteratively reformulating the original planning task to forbid exactly the solution they acquire at each iteration. The solution at each iteration is added to the solution set until extracting exactly $k$ plans.

In this work we made use of the diverse planner \cite{katz2019reshaping}, which uses a top-$k$ planner to generate the top-$k$ best dissimilar solutions to the planning task at hand. As part of this work, we have extended this planner to handle the Temporal Planning setting.
 

\paragraph{} We are now ready to take a closer look at TPNs and their previous usages in the literature.
\section{TPNs \& Pike}
\label{background: tpns and Pike}

A Temporal Planning Network (TPN) is a formalism for representing flexible plans with choices. A TPN is an extension to the Simple Temporal Network \cite{dechter1991temporal}, which adds decision nodes and labels on constraints conditioned on these decisions (also referred to as choices). 
We shall build upon (but simplify) the definition supplied in \cite{levine2018watching} for a Temporal Planning Network under Uncertainty (TPNU), as this is the formalism Pike expects as input. The main difference between a TPN and a TPNU being the mapping of the choice variables to groups of controllable and uncontrollable choices, i.e. which choices does Pike initiate and which does the environment. Formally a TPNU is a tuple $\langle V, \mathcal{E}, C, A\rangle$, where:
% \vspace{-0.5mm}
\begin{itemize}
    \item $V$: The set of decision variables.
    Decision variables are partitioned into two groups $V = V_C \cup V_U$.
    Each $v \in V$ is a discrete variable with a finite domain $Domain(v)$. $V_C$ are \textit{controllable}
    choices, decidable by the executive at run time. $V_U$ are the \textit{uncontrollable} choice variables, whose decisions are determined by the human or nature, rather than the executive.
    
    \item $\mathcal{E}$: The set of notable time points (Events).
    Each $e \in \mathcal{E}$ is associated with a conjunction of choice variable assignments $\varphi_e$. 
    Events can be seen as correlated to the underlying PDDL states of the original task.
    
    \item $C$: The set of temporal constraints. Each $c \in C$ is a tuple $\langle e_s, e_f, l, u, \varphi_c \rangle$ where $e_s$ is the \textit{start} event, $e_f$ is the \textit{finish} event, $\varphi_c$ is a conjunction of choice variable assignments and $l,u \in \mathbb{R}$  represent temporal upper and lower bounds s.t. $\varphi_c \implies (l\leq e_f - e_s \leq u)$. 
    
    \item $\mathbb{A}$: The set of activities. An activity $a \in \mathbb{A}$ is a tuple $\langle c,\alpha \rangle$ where $c \in C$ is a temporal constraint, and $\alpha$ is an action that will be executed online. With $c = \langle e_s, e_f, l, u, \varphi_c \rangle$, action $\alpha$ starts when $e_s$ is scheduled, and terminates when $e_f$ is scheduled. We require that $l >0$
    Activities are related to the durative actions of the underlying PDDL domain.
    
\end{itemize}
We note that in this work we only generate TPNs, i.e. there is no assignment of choices to the uncontrollable set. The task of determining which choices are uncontrollable is a subject for future work.

\subsection*{Pike}
\label{intro: Pike}

Pike \cite{levine2018watching} is an online plan executive for human-robot teamwork that quickly adapts and infers intent based on the preconditions of actions in the plan, temporal constraints, unanticipated disturbances, and choices made previously (by either robot or human). It achieves plan recognition and adaptation concurrently through a single set of algorithms.
Pike takes as input a flexible plan with choices defined using the TPNU formalism. During execution, Pike makes use of a state estimator and activity recognition module for additional inputs in order to access and respond to state disturbances and recognize
the choices made by the human counterpart. Using these inputs Pike infers his team-mate's intent by reasoning about the uncontrollable choices, and makes the controllable choices such that the end-goal will be achieved.
Using Pike to control robots results in a mixed initiative execution in which humans and robots simultaneously work and adapt to each other to accomplish a task.\\
In order to dispatch activities at their proper times, which
may also depend on the choices made, Pike leverages ideas
originally presented in the Drake executive \cite{conrad2010flexible}.
A labeled all-pairs shortest path (APSP) is computed on
the TPNU, to determine which events occur before which
other events. This labeled APSP is used as a dispatchable form for execution.


\paragraph{} To make sure our process provides Pike with high quality TPN, we will want some optimization formalism to use during the generation phase. 
\section{Constraint Optimization}
\label{background: COP}
A Constraint Optimization Problem (COP) is a mathematical optimization problem defined by an objective function with respect to some variables in the presence of constraints on those variables. 
These constraints can either be strictly required (hard constraints) or have variable range values for which to be penalized to an extent based on the conditions that are not satisfied (soft constraints).

A solution to a COP is a mapping from each variable to a value, selected from a finite domain, s.t. the objective functions achieves its desired value, be it minimal for such problems as reducing cost or energy usage, or maximal for accumulating reward or raising utility.
The constraints of the problem limit the variable values that can be assigned simultaneously.

Formally a COP is a tuple $\langle X,D,C,O\rangle$: 
\begin{itemize}
    \item $X$: The set of variables.  
    \item $D$: The set of their respective domains of values. 
    \item $C$: The set of constraints.
    \item $O$: The objective function.
\end{itemize}

% \begin{table}[ht!]
% \centering
% \begin{tabularx}{\textwidth}{ c  X }

% \textbf{Var}  & \textbf{Definition} \\
% \hline
% \\
% $X$:     & \ {The set of variables.}  \\
% \\
% $D$:     & {The set of their respective domains of values. }       \\
% \\
% $C$:     & {The set of constraints.}      \\
% \\
% $O$:     & {The objective function.}      \\
% \\

% \end{tabularx}
% \caption{Constraint Optimization Problem}
% \label{tab:COP}
% \end{table}

Further on, in chapter \ref{chap: generating TPNS},  when constructing a TPN out of the diverse solutions we obtain, we will model the question of which time-points to merge between plans, such as to maximize our TPN's utility, as a COP.

% In this work we have made use of the MiniZinc \cite{nethercote2007minizinc} formalism while solving the COP running the Geocode \cite{gecode} solver.