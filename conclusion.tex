\chapter{Final Remarks \& Open Questions}
\label{chap:conclusion}

\section{Final Remarks}
We have presented the first approach for automatically generating Temporal Planning 
Networks from a description of a planning task. This makes the useful tools based 
on the TPN formalism, such as the Pike executive \cite{levine2018watching}, much more
broadly applicable, as there is no need to manually generate a TPN or RMPL 
program \cite{kim2001executing}. We have also 
adapted the plan reformulation elimination \cite{katz2018novel} to the temporal
setting, thus creating the first diverse temporal planner.

\section{Open Questions}
In this paper, we focused on fully controllable TPNs, that is we assume all the
decisions are ours to make as opposed to choices made by the environment. 
In future work, we will address the uncertainty inherent in some domains, 
such as human-robot teamwork -- one of the original motivations for the TPN formalism.
In order to do this, we intend to use a multi-agent formalism, such as 
{\sc MA-STRIPS} \cite{DBLP:conf/aips/BrafmanD08}, and define which agents are 
under our control and which are not. The objective here will be to generate a TPNU.


Additionally, the objective we optimized here, minimizing the size of the resulting TPN,
is only one possible objective. In the context of human-robot teamwork, one may want
to optimize for some human-focused metrics, such as 
the ease of explaining the TPN to the human, the mental effort needed to keep track
of the current state of execution, and the flexibility given to the human at any given
moment during execution.

Finally, as previously mentioned, TPNUs can serve as a middle ground between
contingent planning and conformant planning or naive replanning. 
We intend to explore using such automatically generated TPNUs in planning under 
uncertainty, where a TPNU is generated, and executed until something outside its
specification occurs, when a new TPNU is synthesized.


%In addition, we intend to investigate the possibility of generating {\em recommended} solutions based on the output TPN. Such solutions will take into account the structure of the TPN for determining certain attributes. For example the TPN path with the most possible valid solutions branching out of its time points can be recommended as the {\em safest} solution, supplying more flexibility for uncontrollable situations.

%In addition we intend to investigate ways of boosting the initial diversity between solution obtained by the diverse planner, while still merging a small number of solutions. This we intend to do by obtaining $n>k$ solutions and merging only the $k$ most diverse as deciphered by various distance metrics explored in literature.
%Finally, some effort should be invested in exploring the high memory consumption for larger $k$ values and ways to overcome it.
