\subsection{Diverse Planning}
\label{background: Diverse}
Diverse planning is a method for obtaining multiple dissimilar solutions to the same planning task.
 It is commonly used in decision support scenarios where a human analyst needs to understand the space 
of possible plans or has difficulty specifying the planning domain model that exactly matches their application.
The diversity between plans in a plan set $\Lambda$ is measured by the average distance between them.
Previous literature such as \cite{bryce2014landmark} discussed how to best take advantage of plan characteristics 
to compare and measure this distance using domain-independent criteria such as action-sets, states, causal links and landmarks.

% such as shown in Table \ref{tab:diversity measures}.

% \begin{table}[h!]
% \centering
% \begin{tabularx}{\textwidth}{ c  X }

% \textbf{Criteria}  & \textbf{Definition} \\
% \hline
% \textit{Action Sets} & Actions that are present in the two plans.\\
% \\
% \textit{States} & The behaviors resulting from the execution of the plans (where the behavior is captured in terms of the sequence of states the agent goes through).\\
% \\
% \textit{Causal Links} & The causal structures of the two plans measured in terms of the causal links representing how actions contribute to the goals being achieved. \\
% \\
% \textit{Landmarks} & Atomic propositions or disjunctive sets of propositions that must be satisfied by all plans solving a planning instance.\\

% \end{tabularx}
% \caption{Plan Diversity Measures}
% \label{tab:diversity measures}
% \end{table}

Each such criteria has its own distance metric $D(p,p')$ to calculate the dissimilarity between two given plans $p,p'$. 

\begin{tcolorbox}[colback=blue!5!white,colframe=blue!75!black]
For instance, Given two plans $p,p'$ where each plan consists of a sequence of actions,
we can evaluate these to obtain a sequence of states $s_0,s_1..s_n \& s_0',s_1'..s_m$ for each plan. 
The distance between a pair of states can then be computed as: 
\[S(s,s') = \frac{\mid s \Delta s' \mid}{\mid s \cup s' \mid}\]
Where $\Delta$ is the symmetric difference operator.
To account for difference in plan length, two methods are suggested.
The first method assumes $k \leq k'$ and each state $s_{k'+1}..s_k$ is considered to contribute maximally (i.e one unit) 
to the difference between the plans:
\[D_{S}(p,p') = \frac{1}{k} \times \bigg[\sum^k_{i=1} S(s_i,s'_i) + k'-k\bigg]\]
The second method assumes the shorter plan simply stays in the goal state $s_g$ for the rest of the duration of the longer plan:
\[D_{S}(p,p') = \frac{1}{k} \times \bigg[\sum^{k-1}_{i=1} S(s_i,s'_i) + \sum^{k'}_{i=k} S(s_k,s'_i)\bigg] \]

\end{tcolorbox}


\paragraph{The Maximal Plan Diversity Problem}
\textit{Given a plan distance metric $D$, a number $k$ and a planning problem instance $P$, obtain a collection of $k$ solution plans $p_1, p_2...p_k \in \Lambda$ solving $P$, such that no other set of $k$ solution plans $\Lambda'$ solving $P$ upholds $Div(\Lambda) < Div(\Lambda')$ } 
\\
Finding the maximum set can be impossible depending on how $D$ is defined\cite{Coman2011GeneratingMetrics}. Therefore, a relaxation of this problem is solved instead, where a distance metric D is taken into account during plan set generation, but maximal plan diversity is not guaranteed.

    
\paragraph{Top-k Planning}
Top-$k$ planning is the task of obtaining a set of $k$ solutions s.t. there exists no better quality solution outside that set. This approach values the quality of the solutions over their diversity from one another.
The iterative plan Forbid Reformulation \cite{katz2018novel} approach to top-$k$ planning
exploits cost-optimal planners by iteratively reformulating the original planning task to forbid exactly the solution they acquire at each iteration. The solution at each iteration is added to the solution set until extracting exactly $k$ plans.

In this work we made use of the diverse planner \cite{katz2019reshaping}, which uses a top-$k$ planner to generate the top-$k$ best dissimilar solutions to the planning task at hand. As part of this work, we have extended this planner to handle the Temporal Planning setting.
 