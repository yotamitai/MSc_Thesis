\subsection{Merge Selection}
\label{generating: merge selection}
Given a naive TPN and its corresponding $\MergeSet$, our algorithm must now decide which 
merge operations to initiate and at what order. 

\begin{tcolorbox}[colback=blue!5!white,colframe=blue!75!black]
     \textbf{Example:} Consider the following configuration --- the naive TPN contains three
     TPSs $\pi_1,\pi_2,\pi_3$ and there exist time points 
     $e_1, e_2, e_3$ in $\pi_1, \pi_2, \pi_3$, respectively, such that $e_1$ is fully
     compatible with $e_2$ \textit{or} $e_3$ but not with both (i.e $m(e_2,e_3) \not \in \MergeSet$).
     Therefore we must choose to either merge $e_1$ to $e_2$ or to merge $e_1$ to $e_3$.
  \end{tcolorbox}


The portrayed scenario demonstrates the dilemma we now face --- 
which pairs of time points to merge? 


%% CP
\subsubsection{Constraint Programming} To solve this optimization problem we call upon
CP to maximize the number of merges we can apply to the naive TPN and formulate the 
problem as a COP. \\ 

We define the set of all IHs participating in the optimization as $N$, in other words,
these are all the {\em different} IHs appearing in $\MergeSet$.
To mark a pair of IHs $a_i,a_j \in N$ as chosen to be merged together, 
we define the Boolean decision variable $m_{i,j}$. The set of all such variables is then
 $M \equiv \{m_{i,j} \mid \forall i,j \in N\}$.

Our objective is to create the smallest possible TPN, and this is done by merging {\em groups} of IHs. Thus, to formulate our optimization objective using the $m_{i,j}$ decision variables, we must create auxiliary decision variables for keeping track of groups. For example, merging $a_1, a_2$ and $a_3$ sets six decision variables to TRUE ($m_{1,2}, m_{2,1}, m_{1,3}, m_{3,1}, m_{2,3}$ and $m_{3,2})$, but only counts as one group.
To do so we assign each IH the shared minimal index $i$ of its group.
For example, given the group containing IHs $a_1,a_2,a_3$, each IH is assigned the shared minimal index $1$.
This assignment is described via an additional decision variable $s_i$ of type integer which is simply derived from the $m_{i,j}$ decision variables. The set of all such variables is then $S \equiv \{s_i \mid \forall i \in N\}$
The initial value for each shared minimal index variable is its own index: $s_i = i$.
The COP therefore contains $N^2 + N$ decision variables.

%% constraints
Let us now formulate the constraints applied in our COP.
These will also define the merging transitivity, as discussed previously, which we wish to apply to the TPN. 
The following constraints hold for both Strict and Loose configurations:
\begin{itemize}
    \item\textbf{Compatible Merges}:  $m_{i, j} \implies \{a_i, a_j\} \in \MergeSet$.
    \item \textbf{Symmetric Merges}:  $m_{i, j} \iff m_{j, i}$.
    \item \textbf{Shared Minimum Node}:  $s_i \neq i \implies m_{i, s_i} \;$ and\\
    $m_{i, j} \implies s_i = min(s_i, s_j)$
\end{itemize}

The Strict configuration contains a single additional constraint dictating that any merging between three time points must uphold that they are all compatible with one another:
 $m_{i, j} \wedge m_{i, k} \implies \{a_i, a_j\},\{a_i, a_k\},\{a_j, a_k\} \in \MergeSet$.
    
The Loose configuration allows more freedom for applying merges but must make sure no two IHs originating from the same original solution are merged. Therefore, for this configuration, we pass $P$, a mapping from time points to original solutions, as input to the COP. Therefore, $P \equiv \{p_i \mid  i = 1,...,k \}$ where $k$ is the number of TPSs. The additional constraints are then:  
\textit{i)} $m_{i, j} \implies p_i \neq p_j$ \textit{ii)} $m_{i, j} \wedge m_{i, k} \implies p_j \neq p_k $ \textit{iii)} $ s_i = s_j \implies p_i \neq p_j$

% the objective
Lastly, the objective function of the COP is to maximize the number of elements in $S$ whose value differs from their index, in other words we want to count the number of IHs which were merged. Formally: 
$f = Max \sum_{i=0}^{N} \mathbbm{1} \;|\;s_i \neq i$